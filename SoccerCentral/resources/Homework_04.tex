\documentclass[11pt]{article} % use larger type; default would be 10pt
\usepackage[utf8]{inputenc} % set input encoding (not needed with XeLaTeX)
\usepackage[margin=1in]{geometry} % to change the page dimensions
\geometry{a4paper} % or letterpaper (US) or a5paper or....

%%% HEADERS & FOOTERS
\usepackage{fancyhdr} % This should be set AFTER setting up the page geometry
\pagestyle{fancy} % options: empty , plain , fancy
\lhead{}\chead{}\rhead{}
\lfoot{}\cfoot{}\rfoot{}

%%% Packages
\usepackage{tabularx} %Better Tables
\usepackage{amsmath} %Better Tables
\usepackage{graphicx}

%%% CUSTOM COMMANDS
% To define a Problem
\usepackage{etoolbox} %Counter.
\newcounter{problem}
\newcommand*{\problem}[2]{\vspace{.5cm}\textbf{\Large{Problem \addtocounter{problem}{1}\arabic{problem} [#1]}} \hfill Topic: \emph{#2}\\\\\indent}
% To define a 8 bit box area
\newcolumntype{b}{>{\hsize=.5\hsize\centering}X}
\DeclareListParser*{\iterateparticipants}{,}
\newcommand*{\eightbit}[8]{
 	{\LARGE
		\begin{tabularx}{.6\textwidth}{| b | b | b | b | b | b | b | b | } \hline
			#1 & #2 & #3 & #4 & #5 & #6 & #7 & #8 \\\hline
		\end{tabularx}
	}
}

\newcommand{\tab}{\hspace*{2em}}

\usepackage{listings}
\usepackage{color}

\definecolor{dkgreen}{rgb}{0,0.6,0}
\definecolor{gray}{rgb}{0.5,0.5,0.5}
\definecolor{mauve}{rgb}{0.58,0,0.82}

\lstset{frame=tb,
  language=Java,
  aboveskip=3mm,
  belowskip=3mm,
  showstringspaces=false,
  columns=flexible,
  basicstyle={\small\ttfamily},
  numbers=none,
  numberstyle=\tiny\color{gray},
  keywordstyle=\color{blue},
  commentstyle=\color{dkgreen},
  stringstyle=\color{mauve},
  breaklines=true,
  breakatwhitespace=true,
  tabsize=3
}


\newcommand{\comment}[1]{\emph{\textcolor{red}{[#1]}}}

\begin{document}

\lhead{Due: October 10, 2016}
\rhead{\bfseries Name (Last, First):\tab\tab\tab\tab}

\begin{center}
	\Large{
		\textbf{
			CSCE 230, Fall 2016\\ %
 			Homework 4
		}
	}
\end{center}


\noindent\textbf{\emph{Notes:}}
	\begin{itemize}
		\item  {\huge\textbf{Read all of the instructions.}}
		\item 	\textbf{This assignment must be typed.} Assignments which are not typed will not be graded.
		\item 	A \underline{hard copy} \textbf{and} an \underline{electronic copy} must be submitted by the beginning of class on the due date.
		\item   The electronic copy must be named as follows: \\ \tab homework\_04\_lastname\_firstname.pdf
		\item 	Staple this coversheet to your completed assignment.
		
	\end{itemize}\vspace{.5cm}

\noindent\textbf{\emph{Material Covered:}}
	\begin{itemize}
		\item[] Appendix A
	\end{itemize}\vspace{.5cm}

\noindent\textbf{\emph{Rubric:}}\\
\begin{center}
	{\renewcommand{\arraystretch}{2.75}
		\begin{tabularx}{\textwidth}{|>{\centering\huge\arraybackslash}X|>{\centering\huge\arraybackslash}X|>{\centering\huge\arraybackslash}X|>{\centering\arraybackslash}X|} \hline
			{\LARGE Problem Number} & {\LARGE Possible Points} & {\LARGE Points Earned} \\ \hline
			1 & 20 & \\\hline
			2 & 20 & \\\hline
			3 & 20 & \\\hline
			4 & 20 & \\\hline
			5 & 20 & \\\hline

			\multicolumn{2}{r|}{\huge{Total:}} & \\\cline{3-3}
	\end{tabularx}}
\end{center}

%Reset header style
\newpage\renewcommand{\headrulewidth}{0pt}\rhead{Homework 4}\chead{CSCE230 Spring 2016}

%Note: For these problems you can use \bar{} to define a variable as its inverse. For multiple variables, such as in part B, \overline{} because \bar{} only makes a line the width of approximately one character.  

\problem{20}{Logic Functions}
Prove the following equivalences by  \textbf{using boolean algebraic manipulation and also by using truth tables.} If any of the following are not equivalent, demonstrate how this was found. \textbf{All work must be shown}, including boolean axioms used.

\begin{enumerate}


\item[a)] $\overline{x_1 \bigoplus x_2} \bigoplus x_3 = \bar{x_1}\bar{x_2}\bar{x_3} + x_1x_2\bar{x_3} + \bar{x_1}x_2x_3 + x_1\bar{x_2}x_3$



\item[b)] $\bar{x_2}x_1 + x_3\bar{x_2} + \bar{x_1}x_3 = \bar{x_2}x_1 + x_3\bar{x_1} $



\end{enumerate}\vspace{-1em}

\problem{20}{Logic Functions}

Consider the logic expressions:\\
$f = x_1 \bar{x_2} \bar{x_5} + \bar{x_1} \bar{x_2 } \bar{x_4 } \bar{x_5} + x_1 x_2 x_4 x_5 + \bar{x_1 } \bar{x_2 } x_3 \bar{x_4} + x_1 \bar{x_2} x_3 x_5 + \bar{x_2 } \bar{x_3 } x_4 \bar{x_5} + x_1 x_2 x_3 x_4 \bar{x_5}$ \\
$g = \bar{x_2 } x_3 \bar{x_4} + \bar{x_2 } \bar{x3} \bar{x_4 } \bar{x_5}  + x_1 x_3 x_4 \bar{x_5} + x_1 \bar{x_2} x_4 \bar{x_5} + x_1 x_3 x_4 x_5 +  \bar{x_1} \bar{x_2 } \bar{x_3 } \bar{x_5}+ x_1 x_2 \bar{ x_3} x_4 x_5$ \\
Prove or disprove using Boolean algebraic manipulations (\underline{no} truth table or Karnaugh maps) that, $ f = g$

\problem{20}{Karnaugh Maps}
Create a truth table for each of the following, then use Karnaugh Maps to minimize the functions

\begin{enumerate}
\item[a)] $f(x_1,x_2,x_3) = \bar{x_1}x_2x_3 + \bar{x_1}x_2\bar{x_3} + \bar{x_1}\bar{x_2}x_3 + x_1x_2\bar{x_3}$
\item[b)] $f(x_1,x_2,x_3,x_4) = \bar{x_1}\bar{x_2}\bar{x_3}x_4 + \bar{x_1}x_2\bar{x_3}\bar{x_4} + \bar{x_1}x_2\bar{x_3}x_4 + \bar{x_1}\bar{x_2}x_3\bar{x_4} + x_1\bar{x_2}\bar{x_3}\bar{x_4} + x_1\bar{x_2}x_3\bar{x_4} + x_1x_2\bar{x_3}\bar{x_4} + x_1x_2x_3\bar{x_4} + x_1x_2x_3\bar{x_4}$
\end{enumerate}\vspace{-1em}

\problem{20}{Logic Circuits}
For the minimized functions in Problem 3, design the circuits using gate symbols. \\\textbf{\textit{Note:}} for this problem you can use a circuit design tool such as Quartus, or draw your circuit. In both cases, include an image of your circuit in your document. 

\newpage

\problem{20}{Logic Design}

Suppose a room has three doors and one bulb in the middle. Beside, each door there is a switch for controlling the bulb to turn ON or OFF. Now, the bulb is controlled by those three switches according to the following logic:

\begin{itemize}
    \item When an odd number of switches are closed (i.e. logically HIGH, or 1), the bulb is ON.
    \item In the remaining cases, the bulb is OFF.
\end{itemize}

Design a logic circuit that can provide this functionality. Show the truth table of the logic design.


\end{document}
